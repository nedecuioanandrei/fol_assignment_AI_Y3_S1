\documentclass[a4paper,12pt]{report}
\usepackage{algorithmic}
\usepackage[linesnumbered,ruled,vlined]{algorithm2e}
\usepackage[margin=2cm]{geometry}
\usepackage[utf8]{inputenc}
\usepackage{listings} 
\usepackage{graphicx} 
\usepackage{color}
\usepackage{xcolor}
\usepackage{hyperref}
\usepackage{verbatim}
%\usepackage{mdframed}

\newcommand{\currentdata}{14 February 2015}
\newtheorem{example}{Example}

\begin{document}
\vspace{-5cm}
\begin{center}
Department of Computer Science\\
Technical University of Cluj-Napoca\\
\includegraphics[width=10cm]{fig/footer}
\end{center}
\vspace{1cm}
%\maketitle
\begin{center}
\begin{Large}
 \textbf{Artificial Intelligence}\\
\end{Large}
\textit{Laboratory activity}\\
\vspace{3cm}
Meze Razvan-Gabriel \& Nedelcu Ioan-Andrei\\
Group: 2\\
Email: tatarazvi@gmail.com anedelcu0226@gmail.com\\
\vspace{12cm}
Teaching Assistant: Adrian Groza\\
Adrian.Groza@cs.utcluj.ro\\
\vspace{1cm}
\includegraphics[width=10cm]{fig/footer}
\end{center}

\tableofcontents

% \input{policy}

%\chapter{Laboratory works}

\chapter{Logics}
\section{Description of the Input File}

The prisoner's dilemma game is described using logical formulas. The game consists of two prisoners, and each prisoner has the choice to confess (C1 and C2) or not confess (-C1 and -C2). Each choice leads to a specific outcome for each prisoner, represented by the variables LS1, MS1, SS1, NS1 for prisoner 1 and LS2, MS2, SS2, NS2 for prisoner 2, which stand for large sentence, medium sentence, small sentence, and no sentence, respectively.

The file also includes formulas that describe the alternative strategies that each prisoner can choose, represented by the variables alternative\_strategy\_C1, alternative\_strategy\_not\_C1, alternative\_strategy\_C2, and alternative\_strategy\_not\_C2, which stand for prisoner 1 switching to confessing or not confessing, and prisoner 2 switching to confessing or not confessing, respectively.

For each alternative strategy, there is an associated alternative sentence for each prisoner, represented by the variables LAS1, MAS1, SAS1, NAS1 for prisoner 1 and LAS2, MAS2, SAS2, NAS2 for prisoner 2.

The file also includes formulas that describe the benefits that each prisoner can gain by switching strategies, represented by the variables P1\_change\_benefit and P2\_change\_benefit, which stand for prisoner 1 and prisoner 2 benefiting, respectively, if they switch strategies from the current state and the other prisoner keeps the same strategy.

Finally, the file includes a formula for the concept of Nash equilibrium, represented by the variable nash\_equilibrium, which indicates that the current state is a Nash equilibrium if no prisoner gets a smaller sentence if he switches strategies and the other prisoner keeps the same strategy.

The goal of the MACE4 input file is to find all the models in which a Nash equilibrium has been reached, in this case, a unique model in which both prisoners confess.

The input file for MACE4 describes the prisoner's dilemma game, in which two prisoners must decide whether to confess or not confess. The game is described using logical formulas.

\subsection{Variables}

The following variables are used in the input file:

\begin{itemize}
\item $C1$: prisoner 1 confesses
\item $C2$: prisoner 2 confesses
\item $LS1$: prisoner 1 gets a large sentence
\item $MS1$: prisoner 1 gets a medium sentence
\item $SS1$: prisoner 1 gets a small sentence
\item $NS1$: prisoner 1 gets no sentence
\item $LS2$: prisoner 2 gets a large sentence
\item $MS2$: prisoner 2 gets a medium sentence
\item $SS2$: prisoner 2 gets a small sentence
\item $NS2$: prisoner 2 gets no sentence
\end{itemize}

\subsection{Logical Formulas}

The input file includes the following logical formulas:

\begin{itemize}
\item If both prisoners do not confess, then both prisoners get small sentences: \
$(-C1 \land -C2) \leftrightarrow (-LS1 \land -LS2 \land -MS1 \land -MS2 \land SS1 \land SS2 \land -NS1 \land -NS2)$
\item If prisoner 1 does not confess and prisoner 2 confesses, then prisoner 1 gets a large sentence and prisoner 2 gets no sentence: \
$(-C1 \land C2) \leftrightarrow (LS1 \land -LS2 \land -MS1 \land -MS2 \land -SS1 \land -SS2 \land -NS1 \land NS2)$
\item If prisoner 1 confesses and prisoner 2 does not confess, then prisoner 1 gets no sentence and prisoner 2 gets a large sentence: \
$(C1 \land -C2) \leftrightarrow (-LS1 \land LS2 \land -MS1 \land -MS2 \land -SS1 \land -SS2 \land NS1 \land -NS2)$
\item If both prisoners confess, then both prisoners get medium sentences: \
$(C1 \land C2) \leftrightarrow (-LS1 \land -LS2 \land MS1 \land MS2 \land -SS1 \land -SS2 \land -NS1 \land -NS2)$
\item $alternative\_strategy\_C1$: prisoner 1 can switch his decision to confessing, from the current state
\item $alternative\_strategy\_not\_C1$: prisoner 1 can switch his decision to not confessing, from the current state
\item $alternative\_strategy\_C2$: prisoner 2 can switch his decision to confessing, from the current state
\item $alternative\_strategy\_not\_C2$: prisoner 2 can switch his decision to not confessing, from the current state
\item $LAS1$: prisoner 1 gets a large sentence if he switches to an alternative strategy
\item $MAS1$: prisoner 1 gets a medium sentence if he switches to an alternative strategy
\item $SAS1$: prisoner 1 gets a small sentence if he switches to an alternative strategy
\item $NAS1$: prisoner 1 gets no sentence if he switches to an alternative strategy
\item $LAS2$: prisoner 2 gets a large sentence if he switches to an alternative strategy
\item $MAS2$: prisoner 2 gets a medium sentence if he switches to an alternative strategy
\item $SAS2$: prisoner 2 gets a small sentence if he switches to an alternative strategy
\item $NAS2$: prisoner 2 gets no sentence if he switches to an alternative strategy
\end{itemize}



The input file includes the following logical formulas, starting from the line "% each pair of desicions...":

\begin{itemize}
\item Each strategy has an alternative:
\begin{itemize}
\item $C1 \leftrightarrow alternative\_strategy\_not\_C1$
\item $C1 \leftrightarrow -alternative\_strategy\_C1$
\item $-C1 \leftrightarrow alternative\_strategy\_C1$
\item $-C1 \leftrightarrow -alternative\_strategy\_not\_C1$
\item $C2 \leftrightarrow alternative\_strategy\_not\_C2$
\item $C2 \leftrightarrow -alternative\_strategy\_C2$
\item $-C2 \leftrightarrow alternative\_strategy\_C2$
\item $-C2 \leftrightarrow -alternative\_strategy\_not\_C2$
\end{itemize}
\item Each alternative strategy, if applied, for each prisoner, has an alternative sentence associated with it (considering that the other prisoner has the same strategy):
\begin{itemize}
\item $alternative\_strategy\_C1 \land C2 \leftrightarrow -LAS1 \land MAS1 \land -SAS1 \land -NAS1$
\item $alternative\_strategy\_not$
\item $alternative\_strategy\_not\_C1 \land C2 \leftrightarrow LAS1 \land -MAS1 \land -SAS1 \land -NAS1$
\item $alternative\_strategy\_C1 \land -C2 \leftrightarrow -LAS1 \land -MAS1 \land -SAS1 \land NAS1$
\item $alternative\_strategy\_not\_C1 \land -C2 \leftrightarrow -LAS1 \land -MAS1 \land SAS1 \land -NAS1$
\item $C1 \land alternative\_strategy\_C2 \leftrightarrow -LAS2 \land MAS2 \land -SAS2 \land -NAS2$
\item $C1 \land alternative\_strategy\_not\_C2 \leftrightarrow LAS2 \land -MAS2 \land -SAS2 \land -NAS2$
\item $-C1 \land alternative\_strategy\_C2 \leftrightarrow -LAS2 \land -MAS2 \land -SAS2 \land NAS2$
\item $-C1 \land alternative\_strategy\_not\_C2 \leftrightarrow -LAS2 \land -MAS2 \land SAS2 \land -NAS2$
\end{itemize}
\end{itemize}

\begin{itemize}
\item $P1\_change\_benefit$: prisoner 1 benefits if he switches strategies from current state and the other prisoner keeps the same strategy (prisoner 1 gets a smaller sentence if he switches)
\item $P2\_change\_benefit$: prisoner 2 benefits if he switches strategies from current state and the other prisoner keeps the same strategy (prisoner 2 gets a smaller sentence if he switches)
\end{itemize}

\begin{itemize}
\item For each alternative strategy, there is a condition for which it is advantageous for the prisoner to change strategies:
\begin{itemize}
\item $alternative\_strategy\_C1 \land C2 \land -P1\_change\_benefit$
\item $alternative\_strategy\_not\_C1 \land C2 \land P1\_change\_benefit$
\item $alternative\_strategy\_C1 \land -C2 \land -P1\_change\_benefit$
\item $alternative\_strategy\_not_C1 \land -C2 \land P1\_change\_benefit$
\item $C1 \land alternative\_strategy\_C2 \land -P2\_change\_benefit$
\item $C1 \land alternative\_strategy\_not\_C2 \land P2\_change\_benefit$
\item $-C1 \land alternative\_strategy\_C2 \land -P2\_change\_benefit$
\item $-C1 \land alternative\_strategy\_not_C2 \land P2\_change\_benefit$
\end{itemize}
\end{itemize}

\begin{itemize}
\item For prisoner 1, it is advantageous to change strategies if the alternative sentence is no sentence or a small sentence:
\begin{align*}
& (NAS1 \land LS1) \lor (NAS1 \land MS1) \lor (NAS1 \land SS1) \lor (SAS1 \land LS1) \
& \quad \lor (SAS1 \land MS1) \lor (MAS1 \land LS1) \
& \leftrightarrow P1\_change\_benefit
\end{align*}
\item For prisoner 1, it is not advantageous to change strategies if the alternative sentence is a large sentence or a medium sentence:
\begin{align*}
& (LAS1 \land NS1) \lor (MAS1 \land NS1) \lor (SAS1 \land NS1) \lor (LAS1 \land SS1) \
& \quad \lor (MAS1 \land SS1) \lor (LAS1 \land MS1) \
& \leftrightarrow -P1\_change\_benefit
\end{align*}
\item For prisoner 2, it is advantageous to change strategies if the alternative sentence is no sentence or a small sentence:
\begin{align*}
& (NAS2 \land LS2) \lor (NAS2 \land MS2) \lor (NAS2 \land SS2) \lor (SAS2 \land LS2) \
& \quad \lor (SAS2 \land MS2) \lor (MAS2 \land LS2) \
& \leftrightarrow P2\_change\_benefit
\end{align*}
\item For prisoner 2, it is not advantageous to change strategies if the alternative sentence is a large sentence or a medium sentence:

\begin{align*}
& (LAS2 \land NS2) \lor (MAS2 \land NS2) \lor (SAS2 \land NS2) \lor (LAS2 \land SS2) \
& \quad
\end{align*}
\item For prisoner 2, it is not advantageous to change strategies if the alternative sentence is a large sentence or a medium sentence:
\begin{align*}
& (LAS2 \land NS2) \lor (MAS2 \land NS2) \lor (SAS2 \land NS2) \lor (LAS2 \land SS2) \
& \quad \lor (MAS2 \land SS2) \lor (LAS2 \land MS2) \
& \leftrightarrow -P2\_change\_benefit
\end{align*}

\item A Nash equilibrium has been reached if both prisoners do not benefit from changing strategies: \
$nash\_equilibrium \leftrightarrow (-P1\_change\_benefit \land -P2\_change\_benefit)$
\end{itemize}

\subsection{Goals}

The input file includes the following goal:

\begin{itemize}
\item Find all states that represent a Nash equilibrium: \
$-nash\_equilibrium$
\end{itemize}

\bibliographystyle{plain}
\bibliography{is}

\appendix

\chapter{Your original code}
Don't be a cheater! Cheating affects your colleagues, scholarships and a lot more.
This section should contain only code developed by you, without any line re-used from other sources. 
This section helps me to correctly evaluate your amount of work and results obtained. 


\vspace{2cm}
\begin{center}
Intelligent Systems Group\\
\includegraphics[width=10cm]{fig/footer}
\end{center}



\end{document}
